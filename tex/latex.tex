\section{Further reading on \LaTeX{}}\label{ref:latex}

\subsection{Tex vs. Word vs. Writer}
This article compares LaTeX with Micorosft's Word and OpenOffice's Writer. First of all it makes a distinction between 'Word processors' and a 'Typesetting environement', where LaTeX falls into the latter. It goes into great detail about how LaTeX is typographically the superior product. The article concludes on the notion that...
\begin{quote}
If the visual output is the decisive criterion, there really is no competition: LaTeX wins on all counts.
\end{quote}
And it very well may be, but that doesn't take away from the fact that LaTeX and the other two products fill a different gap in the market. Whereas LaTeX is very good at its purpose, Word and Writer provide easy integration of spreadsheets, images, tables, etcetera. In general, (as with all the articles below) the main focus of the article seems wrong. For instance, LaTeX provides very granular control over the typesetting, but it is questionable whether a generic user would actually need these featres in day-to-day workflow.  For day-to-day use I would still prefer a simple Word processor over the complete typesetting engine LaTeX offers. However, if my goal was two write an academic report, I would feel more comfortable writing in LaTeX.

\subsection{Word vs. LaTeX}
The second article provides a tabular comparison of Microsoft Word and LaTeX. It is hardly a subjective article as the information is colored and the information is out of date. For example, the article states that that Word lacks bibliography and citation features. However, these features have been included since Microsoft Office 2007. Furthermore the article rips on Words disability to facilitate exchange with foreign programmes.

\begin{quote}
MS Word developers make almost no effort to facilitate exchange with foreign programmes. You may not experience that, because Word is so widespread.
\end{quote}

This claim used to be true, however in modern day and age the compatibility of Word with open document formats has greatly increased. For example, Word nowadays can output the documentin .odf or print directly to .pdf. The platforms on which Word are available remains a problem though, but even that seems to be improving as recently Word got introduced on Android, an alien platform for Microsoft. The learning curve is also rather subjective, but it is generally accepted that Word can, at points, be quirky in its formatting behaviour to say the least. Lastly, the pricing is a given. Whereas LaTeX is available freely, Word is still a paid product. It is up to the enduser whether the pricing is a real issue.

\textit{All things considered, we think that this article is hardly objective. The writer of the article clearly prefers LaTeX and provides his information in a way that makes it almost impossible for Word to win. The notion that Word can only be used to write small pieces of text due to the WYSIWYG editor is clearly false. Many large corporations use Word in daily operation and seem to be doing just fine. That does not take away from the fact that LaTeX is very useful for writing scientific or academic papers whereas this may take more time in Word.}


\subsection{Why should I use LaTeX?}
The question posed on Stack Exchange is whether the OP should change from his tried and tested OpenOffice to LaTeX if he does not write research papers in his dialy workroutine.

The answers pose the following advantages of using LaTeX:
\begin{enumerate}
	\item LaTeX provides the ability to create high quality typographical content. This is especially useful when the document contains a large amount of maths equations or is designed to be printed in high quality.
	\item Formatting is a key point in LaTeX. The structure of a document can theoretically be defined without adding content. The content can be added at any point in time.
	\item (La)TeX has been around for a long time and will be around for the forseeable future. That means that documents written in LaTeX today are stll current thirthy years to come. With other editors this has to be seen.
\end{enumerate}

But there are also some disadvantages or concerns for day-to-day use:
\begin{enumerate}
	\item Picking up LaTeX takes some time to learn. Especially if the end user is familiar with WYSIWYG-type editors like OpenOffice or Word. LaTeX functions differenly at its core. Nevertheless, there are a lot of on- and offline guides available on the topic.
	\item Existing documents written in Word or OpenOffice aren't easily converted over to LaTeX at the click of a button. It may take many hours of manual labour to convert the document.
	\item In LaTeX the content of the document is more important than the design of the document. LaTeX allows for custom designs, but is generally used for uniform text. Readabiliy is key.
\end{enumerate}

A lot of users would be fine with a WYSIWYG editor like OpenOffice (or Word for that matter). It provides the ouput an end user would expect with a rather low learning curve. (La)TeX on the other hand has to clear advantages over OpenOffice, but is more situational. Specific documents require an emphasis on content whereas others need fancy designs. The way we see it, LaTeX is very suited for the academic world but programs like Word and OpenOffice dominate the market in corporate environments.
